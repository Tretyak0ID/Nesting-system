%НАСТРОЙКИ ДОКУМЕНТА
\documentclass[12pt,a4paper]{article} %настройки документа
\usepackage[utf8x]{inputenc} %кодировка
\usepackage{ucs}
\usepackage[english,russian]{babel} %языки
\usepackage[T2A]{fontenc} %шрифт
\usepackage{amsmath, amsfonts, amssymb} %математика

%СКРИПТЫ
\newcommand{\lr}[1]{\left({#1}\right)} %программирование хороших скобочек
\newcommand{\pd}[0]{\partial} %короткая частная производная

%ЗАГОЛОВОК
\author{Третьяк И.Д.}
\title{Математическая модель мелкой воды на вращающейся бипереодической области}


%НАЧАЛО ДОКУМЕНТА
\begin{document}
\maketitle

\begin{abstract}
Построена математическая модель мелкой воды на вращающейся бипереодической области.
\end{abstract}

\section{Математическая модель}

	Математическая модель мелкой воды, используемая для тестирования специальных методов дифференцирования, представляет собой реализацию горизонтальных эффектов динамики атмосферы. Далее дадим краткое описание построенной модели.

\subsection{Непрерывная формулировка задачи}
	
	Дифференциальные уравнения, составляющие непрерывную постановку задачи, рассматривались в двух видах.\\[1mm]
	\textbf{1.} Уравнения мелкой воды в адвективной форме:
	$$\begin{cases}
	\frac{\pd u}{\pd t} = -u\frac{\pd u}{\pd x} - v\frac{\pd u}{\pd y} - g\frac{\pd h}{\pd x} + fv, \\[1mm]
	\frac{\pd v}{\pd t} = -u\frac{\pd v}{\pd x} - v\frac{\pd v}{\pd y} - g\frac{\pd h}{\pd y} - fu, \\[1mm]
	\frac{\pd h}{\pd t} = -\frac{\pd hu}{\pd x} - \frac{\pd hv}{\pd y}
	\end{cases}$$\\[1mm]
	где $u,v$ - компоненты вектора скорости, $h$ - высота уровня жидкости, $g \approx 9.8 \ ms^{-2}$ - ускорение свободного падения, $f \approx 2 \cdot 7.292 \cdot 10^{-5} \ s^{-1}$ - параметр Кориолиса.
	
	В результате реализации схем численного решения данных уравнений, в некоторых тестах, требующих расчета на достаточно большие временные промежутки, мы столкнулись с явлением нелинейной неустойчивости. В результате этого было решено моделировать эквивалентную систему, для которой в аналитической форме выражается закон сохранения энергии $E=h\frac{u^2 + v^2}{2} + h\frac{h^2}{2} = const$.\\[1mm]
	\textbf{2.} Уравнения мелкой воды в векторно-инвариантной форме:
	$$\begin{cases}
	\frac{\pd u}{\pd t} = (\xi + f)v - \frac{\pd}{\pd x}(K+ gh)\\[1mm]
	\frac{\pd v}{\pd t} = -(\xi + f)u + \frac{\pd}{\pd y}(K+ gh)\\[1mm]
	\frac{\pd h}{\pd t} = -\frac{\pd hu}{\pd x} - \frac{\pd hv}{\pd y}
	\end{cases}$$\\[1mm]
	где $\xi = \frac{\pd v}{\pd x} - \frac{\pd u}{\pd y}$ - поле завихренности, $K=\frac{u^2 + v^2}{2}$ - кинетическая энергия.
	
	Данные системы уравнений рассматриваются на двумерной вращающейся бипереодической области. 
	
	Кроме того, на каждом шаге по времени происходит решение диффузионного уравнения, продставляющее собой численный фильтр для сглаживания мелкомасштабных осцилляций:
	$$\frac{\pd \Psi}{\pd t} = -K^2\Delta^2\Psi$$
	где $\Psi$ - сглаживаемое поле, $K$ - коэффициент диффузии, $\Delta$ - оператор Лапласа.
	
\subsection{Дискретная формулировка задачи на регулярной сетке без сгущений}

	Рассмотрим двумерную бипереодическую область $\Omega$, введем на ней регулярную расчетную сетку $\omega \subset \Omega: \ \omega=\{(ih_x, jh_y)\ | \ i=1 \ldots N_x, j=1 \ldots N_y\}$, где $h_x, h_y$ - шаги дискретизации по пространству, $N_x, N_y$ - число интервалов разбиения. Будем решать данные системы уравненний на временном промежутку $[0, T]$, проведем дискретизацию этого промежутку с шагом по времени $\Delta t$, получим $\omega_t = \{k\Delta t \ | \ k=1 \ldots N_t\}$. 
	
	Далее дискретизуем уравнения для получения конечно-разностного аналога в $\omega \times \omega_t$. Для наших целей удобнее всего будет воспользоваться разделением дискретизации уравнений по пространству и по времени.
	
	Рассмотрим переодические разностные операторы дифференцирования по пространству, в рамках модели реализовано четыре вида таких операторов.\\[1mm]
	\textbf{1.} Центральная разность 2го порядка $D_2$ для дискретизации динамических уравнений по пространству:
	$$ \begin{cases}
	   D_2f = \frac{f_{i+1} - f{i-1}}{2h} \ i=1,...,N - 1, \\[1mm]
	   D_2f_{0} = \frac{f_{1} - f_{N - 1}}{2h}, \\[1mm]
	   D_2f_{N} = D_2f_{0}
	\end{cases}$$
	\textbf{2.}   Центральная разность 4го порядка $D_4$ для дискретизации динамических уравнений по пространству:
	$$ \begin{cases}
	   D_4f = \frac{-f_{i+2} + 8f{i+1} - 8f_{i-1} + f_{i-2}}{12h} \ i=2,...,N - 2, \\[1mm]
	   D_4f_{0} = \frac{-f_{2} + 8f{1} - 8f_{N-1} + f_{N-2}}{12h}, \\[1mm]
	   D_4f_{1} = \frac{-f_{3} + 8f{2} - 8f_{0} + f_{N-1}}{12h}, \\[1mm]
	   D_4f_{N-1} = \frac{-f_{1} + 8f{N} - 8f_{N-2} + f_{N-3}}{12h}, \\[1mm]
	   D_4f_{N} = D_4f_{0}
	\end{cases}$$
	\textbf{3.} Дискретизация второй производной второго порядка $D^2_2$ для диффузионного уравнения:
	$$ \begin{cases}
	   D^2_2f = \frac{f_{i+1} - 2f_i + f{i-1}}{h^2} \ i=1,...,N - 1, \\[1mm]
	   D^2_2f_{0} = \frac{f_{1} - 2f_{0} + f_{N - 1}}{h^2}, \\[1mm]
	   D^2_2f_{N} = D^2_2f_{0}
	\end{cases}$$
	\textbf{4.} Дискретизация второй производной второго порядка $D^2_4$ для диффузионного уравнения:
	$$ \begin{cases}
	   D^2_4f = \frac{-f_{i+2} + 16f{i+1} - 30f_i + 16f_{i-1} - f_{i-2}}{12h^2} \ i=2,...,N - 2, \\[1mm]
	   D^2_4f_{0} = \frac{-f_{2} + 16f{1} - 30f_0 + 16f_{N-1} - f_{N-2}}{12h^2}, \\[1mm]
	   D^2_4f_{1} = \frac{-f_{3} + 16f{2} - 30f_1 + 16f_{0} - f_{N-1}}{12h^2}, \\[1mm]
	   D^2_4f_{N-1} = \frac{-f_{1} + 16f{0} - 30f_{N-1} + 16f_{N-2} - f_{N-3}}{12h^2}, \\[1mm]
	   D^2_4f_{N} = D^2_4f_{0}
	\end{cases}$$
	где N - число узлов в соответствующем направлении, h - шаг расчетной сетки.
 
\end{document}