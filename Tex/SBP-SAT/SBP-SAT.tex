%НАСТРОЙКИ ДОКУМЕНТА
\documentclass[12pt,a4paper]{article} %настройки документа
\usepackage[utf8x]{inputenc} %кодировка
\usepackage{ucs}
\usepackage[english,russian]{babel} %языки
\usepackage[T2A]{fontenc} %шрифт
\usepackage{amsmath, amsfonts, amssymb} %математика

%СКРИПТЫ
\newcommand{\lr}[1]{\left({#1}\right)} %программирование хороших скобочек
\newcommand{\pd}[0]{\partial} %короткая частная производная

%ЗАГОЛОВОК
\author{Третьяк И.Д.}
\title{SBP-SAT подход для горизонтальной аппроксимации уравнений динамики атмосферы на сетках с локальным повышением разрешения}


%НАЧАЛО ДОКУМЕНТА
\begin{document}
\maketitle

\begin{abstract}
В рамках математической модели мелкой воды, допускающей локальное повышение разрешения расчетной сетки, реализовано и протестировано семество SBP-SAT (Summation-By-Parts Simultaneous Approximation Terms) меотодов.
\end{abstract}

\section{Семейство SBP-SAT методов}
\subsection{Summation-By-Parts}
\subsection{Simultaneous Approximation Terms}
	
\section{Результаты тестирования}

\end{document}